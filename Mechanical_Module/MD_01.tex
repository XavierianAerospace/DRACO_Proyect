



# Brazo

## Muñeca

### 8 de Agosto del 2025
La idea base es buscar una forma de poder tener movilidad en la muñeca, permitiendo así una mayor versatilidad en los movimientos y acciones que el brazo puede reaalizar. Para esto, se propone un diseño que permita rotación y flexión, similar a una muñeca humana. 
Para poder lograr esto se toma como analisis principal el poder moverse, tener apoyo  y poder tener agarre de objetos con la muñeca.

Se busca que este utilice como minimo dos mecanismos para el agarre para tener la seguridad y confianza de que en caso de que falle uno se tien eel otro como respaldo.
Se tiene la idea tentativa principalmente de hacer con tendones de algun material resistente  y flexible, que permita la movilidad y agarre de objetos. Estos tendones se conectarían a un sistema de poleas o engranajes que permitirían controlar el movimiento de la muñeca de manera precisa.
La otra opcion es por medio de electroneumatica o electohidraulica, que permitiría un control más preciso y rápido del movimiento de la muñeca. esto con el fin de necesitar fuerza y velocidad en el agarre de objetos.

Se tiene pensado hacerlo con un material y con una forma basada en ivestigaciones del profesor Robert Shepherd de la universidad de hardvard, que ha trabajado en el desarrollo de actuadores suaves y flexibles que imitan la musculatura humana. y tienenen un soporte de presion y manejo necesarios segun los papers investigativos que se han revisado


https://scholar.google.com/citations?view_op=list_works&hl=en&hl=en&user=9ba0QeQAAAAJ


https://dash.harvard.edu/server/api/core/bitstreams/7312037d-2f37-6bd4-e053-0100007fdf3b/content

https://dash.harvard.edu/server/api/core/bitstreams/7312037d-3308-6bd4-e053-0100007fdf3b/content

Se avanzo con los sigueintes diseños y se busca tener esta idea pero falta investigar mas sobre estos mecanismos, tipos de fomrmas y demas en esta seccion


Se realizaron estos avances y busquedas y se llego a las sigueintes referencias e imagenes

https://muchoreptil.wordpress.com/2015/01/11/ficha-gecko-leopardo-eublepharis-mascularius/

https://www.google.com/search?udm=2&q=robotica+espacial#vhid=FGqJ8RIfcGL-5M&vssid=mosaic

https://www.youtube.com/watch?v=8Zdj66ljk0I

https://www.google.com/search?udm=2&q=snake+robot+mars#vhid=k_-tTJeT_u5WzM&vssid=mosaic

https://www.youtube.com/watch?v=ifCIDT4X9AM

https://en.wikipedia.org/wiki/Optimus_%28robot%29

https://www.google.com/search?udm=2&q=antenas+aviones+f22#vhid=E70JzQe3rZKxVM&vssid=mosaic

https://www.google.com/search?udm=2&q=ligamentos+robotica#vhid=gA-F883H4qErmM&vssid=mosaic

